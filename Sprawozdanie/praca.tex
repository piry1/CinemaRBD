\documentclass[eng,printmode,oneside,openany]{mgr2} %openany openright
\usepackage{polski} %przydatne podczas składania dokumentów w j. polskim
%\usepackage[polish]{babel}%alternatywnie do pakietu polski, wybrać jeden z nich
%\usepackage[utf8x]{inputenc} %kodowanie znaków, zależne od systemu
%\usepackage{polski} %przydatne podczas składania dokumentów w j. polskim
%\usepackage[cp1250]{inputenc} %kodowanie znaków, zależne od systemu
%\usepackage[T1]{fontenc} %poprawne składanie polskich czcionek
%pakiety do grafiki
\usepackage[utf8]{inputenc} %kodowanie znaków, zależne od systemu
\usepackage[T1]{fontenc} %poprawne składanie polskich czcionek
\usepackage{lmodern}
%\usepackage[utf8x]{inputenc}
%\usepackage{ucs}
%\usepackage[MeX]{polski}
\usepackage{graphicx}
\usepackage{subfigure}
\usepackage{psfrag}
%pakiety dodające dużo dodatkowych poleceń matematycznych
\usepackage{amsmath}
\usepackage{amsfonts}
%pakiety wspomagające i poprawiające składanie tabel
\usepackage{supertabular}
\usepackage{array}
\usepackage{tabularx}
\usepackage{hhline}
\usepackage{listings}

\usepackage{float}
\usepackage{indentfirst}
\usepackage{color}
\usepackage{enumerate}

\usepackage{setspace}
\usepackage{tabularx}
\usepackage{color,calc}
\usepackage{listings}


\usepackage{wrapfig}
\usepackage{lscape}
\usepackage{rotating}
\usepackage{epstopdf}
%\usepackage{soul} % pakiet z komendami do podkreślania tekstu

\usepackage{ebgaramond} % pakiet z czcionkami garamond, potrzebny tylko do strony tytułowej, musi wystąpić przed pakietem tgtermes
%\usepackage[left=2.5cm,right=2.5cm,top=2.5cm,bottom=2.5cm]{geometry}


%% Aby uzyskać polskie literki w pdfie (a nie zlepki) korzystamy z pakietu czcionek tgterms. 
%% W pakiecie tym są zdefiniowane klony czcionek Times o kształtach: normalny, pogrubiony, italic, italic pogrubiony.
%% W pakiecie tym brakuje czcionki o kształcie: slanted (podobny do italic). 
%% Jeśli w dokumencie gdzieś zostanie zastosowana czcionka slanted (np. po użyciu komendy \textsl{}), to
%% latex dokona podstawienia na czcionkę standardową i zgłosi to w ostrzeżeniu (warningu).
%% Ponadto tgtermes to czcionka do tekstu. Wszelkie matematyczne wzory będą sformatowane domyślną czcionką do wzorów.
%% Jeśli wzory mają być sformatowane z wykorzystaniem innych czcionek, trzeba to jawnie zadeklarować.

%% Po zainstalowaniu pakietu tgtermes może będzie trzeba zauktualizować informacje 
%% o dostępnych fontach oraz mapy. Można to zrobić z konsoli (jako administrator)
%% initexmf --admin --update-fndb
%% initexmf --admin --mkmaps

\usepackage{tgtermes} 

%pakiet wypisujący na marginesie etykiety równań i rysunków zdefiniowanych przez \label{}, chcąc wygenerować finalną wersję dokumentu wystarczy usunąć poniższą linię
%\usepackage{showlabels} 

%definicje własnych poleceń
\newcommand{\R}{I\!\!R} %symbol liczb rzeczywistych, działa tylko w trybie matematycznym
\newtheorem{theorem}{Twierdzenie}[section] %nowe otoczenie do składania twierdzeń
\newcommand{\lssetdef}{\lstset{
		backgroundcolor=\color[rgb]{0.9,0.9,0.9},
		basicstyle={\small\ttfamily},
		breaklines=true,
		frame=l,
		tabsize=2,
		basicstyle=\small,
		xleftmargin={0.75cm},
		numbers=left,
		stepnumber=1,
		firstnumber=1,
		numberfirstline=true,
		showspaces=false,                % show spaces everywhere adding particular underscores; it overrides 'showstringspaces'
		showstringspaces=false,          % underline spaces within strings only
		showtabs=false,                  % show tabs within strings adding particular underscores
}}
	
%dane do złożenia strony tytułowej
\title{Rozproszony system bazodanowy przeznaczony do obsługi kina}
\engtitle{}
\author{Radosław Taborski - 209347\\Piotr Konieczny - 209174}
\supervisor{dr inż. Robert Wójcik}
%\guardian{dr hab. inż. Imię Nazwisko Prof. PWr, I-6} %nie używać jeśli opiekun jest tą samą osobą co prowadzący pracę

%\date{2017} %standardowo u dołu strony tytułowej umieszczany jest bieżący rok, to polecenie pozwala wstawić dowolny rok

%poniżej jest lista kierunków i specjalności na wydziale elektroniki, należy wybrać właściwe lub dopisać jeśli nie ma odpowiednich
\field{Informatyka (INF)}
\specialisation{Inżynieria systemów informatycznych (INS)}

%tutaj zaczyna się właściwa treść dokumentu
\begin{document}
	%\bibliographystyle{unsrt} %tylko gdy używamy BibTeXa, ustawia polski styl bibliografii

	\maketitle %polecenie generujące stronę tytułową
	\topmargin -12mm
	%TODO: ten fragment do wersji archiwalnej
	%\newpage 
	%\thispagestyle{empty}
	%\ 
	%\newpage
	\linespread{0.8}	

	\setcounter{page}{2}
	\tableofcontents %spis treści
	\linespread{1}
	%opcjonalnie może się tu pojawić spis rysunków i tabel
	\listoffigures
	\addcontentsline{toc}{chapter}{Spis rysunków} %utworzenie w spisie treści pozycji Spis rysunków
	\listoftables
	\addcontentsline{toc}{chapter}{Spis tabel} %utworzenie w spisie treści pozycji Spis tabel
	\renewcommand\lstlistingname{Listing}
	\renewcommand\lstlistlistingname{Spis listingów}
	
	%\mbox{}\pdfbookmark[0]{Spis listingów}{spisListingow.1}
	\lstlistoflistings
	\addcontentsline{toc}{chapter}{Spis listingów} %utworzenie w spisie treści	pozycji Spis listingów

	\chapter{Wstęp}

\section{Cele projektu}
Celem projektu jest stworzenie systemu wspomagającego obsługę kina w oparciu o rozproszoną i obiektową bazę danych. System będzie umożliwiać zarządzanie kinem – z wykorzystaniem relacyjnych baz danych replikujących miedzy sobą dane. W pojedynczym węźle bazy danych zawarte będą tabele opisujące miedzy innymi – seanse filmowe, przydział ich do poszczególnych sal kinowych. Aplikacja będzie umożliwiać ponadto tworzenie nowych wpisów w zależności od rodzaju użytkownika obsługującego program. Pracownik kina będzie wprowadzać nowe seanse do bazy; podczas bezpośredniej sprzedażny biletów będzie również wykreślał miejsca
na sali już zajęte – miejsca zawarte na biletach, poszukiwanie rezerwacji wykonanej na konkretna osobę (po imieniu lub nazwisku, czy tez numerze rezerwacji). Użytkownik(Klient) będzie mógł rezerwować konkretne miejsce na określony seans.

\section{Założenia projektowe}
Projekt został wykonany przy użyciu MySQL 5.7. Rozproszoność systemu oparta została o dockery, na których skonfigurowane zostały węzły zarówno slave jak i master. W trakcie realizacji projektu zostały wykorzystane mechanizmy replikacji master-slave oraz master-slave z opóźnieniem. Do wykonania projektu bazy danych wykorzystane zostało narzędzie Microsoft Visio. Zarządzanie bazą danych odbywało się z poziomu narzędzia zwanego phpMyAdmin. Aplikacja kliencka została wykonana w technologii webowej, z wykorzystaniem platformy programistycznej Angular2 oraz języka programowania TypeScript. Komunikacja między bazą danych a aplikacją kliencką zapewnia api restowe napisane w języku PHP. 

\section{Zakres projektu}
Zakres projektu dotyczy zaprojektowania i implementacji rozproszonego systemu bazodanowego dla kina. Projekt składa sę z kilku etapów.

\begin{itemize}
	\item Określenie wymagań funkcjonalnych aplikacji bazodanowej
	\item Testowanie mechanizmów replikacji oraz rozpraszania danych
	\item Opracowanie modelu konceptualnego i fizycznego bazy danych
	\item Implementacja bazy danych, procedur i widoków
	\item Projektowanie i implementacja aplikacji klienckiej
	\item Wdrożenie i testowanie aplikacji klienckiej
\end{itemize}


	\chapter{Replikacja w systemie baz danych MySQL}

\section{Pojęcie replikacji i podstawowe informacje}

\section{Replikacja master-slave}



	\chapter{Model konceptualny i fizyczny baz danych}

\section{Model konceptualny}

\begin{figure} [H]
	\centering
	\includegraphics[width=1\linewidth]{rozdzial03/model_konceptualny.png}
	\caption{Model konceptualny węzła rozproszonej bazy danych wykonany w programie Microsoft Visio}
	\label{fig:model_koncepcyjny}
\end{figure}

W bazie danych dominują relacje typu jeden do wielu. Na każdej sali może odbywać się wiele seansów, natomiast każdy seans posiada tylko jeden film i jedną salę. Każde miejsce również ma przypisaną jedną konkretną salę. Na jedno miejsce może przypadać wiele rezerwacji, w zależności od seansu. Również każdy użytkownik może mieć wiele kupionych biletów lub nie mieć ich wcale. Oraz istnieje wielu użytkowników o tych samych uprawnieniach. Jedyną relacją, która nie jest jeden do wielu, to powiązanie między tabelą \textit{Bilety} i \textit{Rezerwacje}. Na każdą rezerwację może być maksymalnie jeden bilet, który odpowiada jednej rezerwacji, rozumianej tutaj jako miejsce na konkretny seans, które może być zajęte lub też nie.


\section{Model fizyczny}

\begin{figure} [H]
	\centering
	\includegraphics[width=0.95\linewidth]{rozdzial03/model_fizyczny.png}
	\caption{Model fizyczny węzła rozproszonej bazy danych wykonany w programie Microsoft Visio}
	\label{fig:model_fizyczny}
\end{figure}

\textbf{Tabele:}
\begin{itemize}
	\item \textit{Filmy} - przechowuje najważniejsze informacje o filmach, takie jak ich tytuł, nazwiska reżyserów, daty premiery, gatunek oraz czas trwania, żadna z tych wartości nie może pozostać pusta;
	\item \textit{Sale}	- jedynymi potrzebnymi w projekcie parametrami charakteryzującymi sale są jej unikalna nazwa i pojemność ukazująca liczbę dostępnych miejsc siedzących;
	\item \textit{Seanse} - każdy seans ma przypisany film oraz salę. Dodatkowo przechowuje takie informacje jak data i godzina wyświetlenia seansu, cenę oraz pozostałą liczbę wolnych miejsc;
	\item \textit{Miejsca} - w tej tabeli przechowywane są wszystkie miejsca siedzące dostępne w kinie. Każde miejsce znajduje się w swojej określonej sali. Każde miejsce dodatkowo ma też numer i rząd w którym się znajduje na sali;
	\item \textit{Rezerwacje} - jest to spis wszystkich miejsc na wszystkie dostępne seanse. Dodatkowo przechowywana jest wartość zero-jedynkowa odpowiadająca za stan czy miejsce jest już zajęte;
	\item \textit{Uprawnienia} - w projekcie przewidziane są uprawniania dwojakiego rodzaju: uprawniania administratora i użytkownika. Informacja ta ma kluczowe znaczenie podczas logowania do systemu i wyświetlania w nim dostępnych funkcjonalności;
	\item \textit{Użytkownicy} - każdy użytkownik jest zobligowany podczas procesu rejestracji do podania takich informacji o sobie jak imię i nazwisko, oraz podania hasła i unikalnego loginu przez który będzie się logował i to właśnie te dane są przechowywane w tej tabeli. Dodatkowo również do każdego użytkownika dodawane są jego uprawnienia: administratora lub zwykłego użytkownika;
	\item \textit{Bilety} - każdy bilet jest przypisany do konkretnego użytkownika i do konkretnego rekordu z tabeli \textit{Rezerwacje}.
\end{itemize}
	\chapter{Implementacja baz danych w środowisku MySQL}

\section{Realizacja bazy danych}

\subsection{Tabele}

Tabele generowano za pomocą skryptów MySQL.

\begin{itemize}
	\item zdefiniowano klucze główne;
	\item zdefiniowano klucze obce;
	\item dodano ograniczenia jeżeli jakiś element nie może być pusty;
	\item tam gdzie to konieczne (login użytkownika ,nazwa sali i nazwa uprawnień) dodano ograniczenia, aby dane te były unikalne w swoich tabelach;
	\item sprawdzanie czy podana liczba należy do przedziału wykorzystując słowo kluczowe \textit{CHECK} (pojemność sali musi się znajdować w przedziale od 20 do 450 miejsc);
	\item zabezpieczenie przed nadpisywaniem już wcześniej utworzonych tabel \textit{IF NOT EXISTS}.
\end{itemize}

\begin{figure} [H]
	\centering
	\includegraphics[width=0.6\linewidth]{rozdzial04/T_Seanse.png}
	\caption{Generowanie tabeli seansów}
	\label{fig:t_seanse}
\end{figure}

\begin{figure} [H]
	\centering
	\includegraphics[width=0.6\linewidth]{rozdzial04/T_Bilety.png}
	\caption{Generowanie tabeli biletów - przykład użycia klucza głównego, kluczy obcych i ograniczenia "not null"}
	\label{fig:t_bilety}
\end{figure}

\begin{figure} [H]
	\centering
	\includegraphics[width=0.5\linewidth]{rozdzial04/T_Sale.png}
	\caption{Generowanie tabeli sal - przykład użycia CHECK}
	\label{fig:t_sale}
\end{figure}

\subsection{Widoki}

\textbf{Utworzono trzy widoki:} %TODO:uaktualnić

\begin{figure} [H]
	\centering
	\includegraphics[width=0.8\linewidth]{rozdzial04/widoki.png}
	\caption{Wygenerowane widoki}
	\label{fig:views}
\end{figure}

\begin{itemize}
	\item \_Seanse\_ - rozszerza tabelę \textit{Seanse} o dodatkowe informacje o filmie, którego dotyczy seans i sali, na której seans zostanie wyświetlony;
	\item \_Miejsca\_ - rozszerza tabelę \textit{Rezerwacje} o informacje takie jak nazwa sali, rząd i numer miejsca.
\end{itemize}

\begin{figure} [H]
	\centering
	\includegraphics[width=0.6\linewidth]{rozdzial04/V_Seanse_.png}
	\caption{Generowanie widoku seansów}
	\label{fig:v_seanse}
\end{figure}

\begin{figure} [H]
	\centering
	\includegraphics[width=0.6\linewidth]{rozdzial04/V_Miejsca_.png}
	\caption{Generowanie widoku Miejsc}
	\label{fig:v_miejsca}
\end{figure}

\subsection{Procedury}

Rysunek \ref{fig:procedury} przedstawia wszystkie procedury jakie zostały zaimplementowane. Natomiast na rysunku \ref{fig:p_DodajSeans} pokazana została jedna przykładowa procedura, która dodając seans dodaje również pulę miejsc równą pojemności sali, w której odbędzie się seans do tabeli \textit{Rezerwacje}.
\begin{figure} [H]
	\centering
	\includegraphics[width=0.7\linewidth]{rozdzial04/Procedury.png}
	\caption{Przykładowa procedura}
	\label{fig:procedury}
\end{figure}

\textbf{Opisy procedur:}
\begin{itemize}
	\item BiletyUsera - wyświetla wszystkie bilety które zostały zakupione przez konkretnego użytkownika;
	\item DodajAdmina - procedura pozwala na dodanie nowego użytkownika o uprawnieniach administratora;
	\item DodajFilm - pozwala na dodanie nowego filmu do filmoteki kina;
	\item DodajSale - pozwala dodać do bazy danych nowej sali kinowej;
	\item DodajSeans - umożliwia dodanie nowego seansu do oferty kina;
	\item DodajUzytkownika - procedura pozwala na dodanie nowego użytkownika o standardowych prawach zwykłego użytkownika;
	\item EdytujFilm - pozwala na zmienienie wszystkich informacji o filmie dostępnych w bazie;
	\item EdytujSale - pozwala na zmianę nazwy sali;
	\item EdytujSeans - pozwala modyfikować datę godzinę i cenę seansu;
	\item KupBilet - tworzy nowy rekord w tabeli Bilety, przypisuje go do konkretnego użytownika, oraz zmienia ilość wolnych miejsc na seansie, a w tabeli Rezerwacje oznacza miejsce jako zajęte;
	\item SprawdzUzytkownika - sprawdza czy użytkownik o podanym loginie i haśle istnieje w systemie;
	\item UsunFilm - usuwa film z filmoteki kina;
	\item UsunRezerwacje - usuwa wszystkie bilet, anuluje transakcję użytkownika, przywraca miejsce jako niezarezerwowane;
	\item UsunSale - usuwa sale i miejsca przypisane do sali, anuluje wszystkie seanse, które miały odbyć się na danej sali, i anuluje wszystkie bilety na te seanse;
	\item UsunSeans - anuluje seans i wszystkie bilety i rezerwacje na niego;
	\item WyswietlBilet - wyświetla informacje o bilecie o podanym id;
	\item WyswietlMiejsca - pokazuje wszystkie wolne miejsca na konkretny seans, korzystając z widoku \_Miejsca\_;
	\item WyswietlSeanse - wyświetla rozszerzone informacje o seansach z podanym filmem, korzystając z widoku \_Seanse\_.
\end{itemize}

\begin{figure} [H]
	\centering
	\includegraphics[width=1\linewidth]{rozdzial04/P_DodajSeans.png}
	\caption{Przykładowa procedura}
	\label{fig:p_DodajSeans}
\end{figure}

\section{Wykorzystanie mechanizmów replikacji master-slave}
	\chapter{Projekt i implementacja aplikacji klienckiej oraz REST API}

\section{Funkcje aplikacji - diagram przypadków użycia}

%TODO: krótki opis

Użytkownik:
\begin{itemize}
	\item tworzenie nowego konta (podanie loginu, hasła itp.);
	\item logowanie;
	\item przeglądanie filmów, seansów, kupionych biletów;
	\item kupowanie biletów.
\end{itemize}
\vspace*{0.5em}
Administrator:
\begin{itemize}
	\item te same funkcjonalności co użytkownik;
	\item dodawanie/usuwanie/edytowanie seansów;
	\item dodawanie/usuwanie/edytowanie filmów;
	\item dodawanie/usuwanie/edytowanie dostępnych sal.
\end{itemize}

\begin{figure} [H]
	\centering
	\includegraphics[width=0.6\linewidth]{rozdzial05/diagram.png}
	\caption{Diagram przypadków użycia}
	\label{fig:schem}
\end{figure}

\section{Realizacja wybranych funkcjonalności aplikacji}

%TODO: napisać

\section{Realizacja REST API}

%TODO: napisać
	\chapter{Wdrożenie i testowanie aplikacji}

W tej części zostały zamieszczone obrazy działania wprowadzonej aplikacji. Aplikacja działa
poprawnie i zgodnie z przewidywaniami. \\

Na rysunkach od \ref{fig:addFilm} do \ref{fig:endAddFilm} zademonstrowane zostało działanie replikacji na przykładzie dodawania filmu do bazy z poziomu aplikacji webowej.\\

Na rysunku \ref{fig:addFilm} wpisane zostały dane przykładowego filmu, a po naciśnięciu przycisku 'Dodaj film' aplikacja poprzez API REST łączy się z węzłem master rozproszonej bazy danych, gdyż wprowadzana będzie modyfikacja, a następnie dodaje dane do tabeli Filmy (rysunek \ref{fig:FilmMaster}) dzięki procedurze SQL 'DodajFilm'. Wykonana procedura zostaje replikowana do wszystkich węzłów Slave. Tabela Filmy dla wybranego węzła Slave na porcie 8083 została pokazana na rysunku \ref{fig:FilmSlave}. Z węzła Slave również odczytywane są dane poprzez aplikację kliencką, a poprawnie odczytane dane z takiego węzła zostały przedstawione na rysunku \ref{fig:endAddFilm}.

\begin{figure} [H]
	\centering
	\includegraphics[width=1\linewidth]{rozdzial06/5.png}
	\caption{Dodawanie nowego filmu poprzez aplikację}
	\label{fig:addFilm}
\end{figure}

\begin{figure} [H]
	\centering
	\includegraphics[width=1\linewidth]{rozdzial06/7.png}
	\caption{Tabela Filmy na węźle master o porcie 8082}
	\label{fig:FilmMaster}
\end{figure}

\begin{figure} [H]
	\centering
	\includegraphics[width=1\linewidth]{rozdzial06/8.png}
	\caption{Tabela Filmy na węźle slave1 o porcie 8083}
	\label{fig:FilmSlave}
\end{figure}

\begin{figure} [H]
	\centering
	\includegraphics[width=1\linewidth]{rozdzial06/6.png}
	\caption{Nowy film wyświetlany w aplikacji}
	\label{fig:endAddFilm}
\end{figure}


	\chapter{Podsumowanie}

Z powodzeniem zaprojektowano oraz zaimplementowano system obsługi bazodanowej kina.
Wszelkie problemy związane z projektem zostały wyjaśniane i wnikliwie konsultowane z prowadzącym
zajęcia projektowe. Mechanizmy wykorzystywane były skrupulatnie prezentowane na zajęciach,
po drobnych korektach wprowadzono je do systemu bazodanowego tak, aby dążyć do
implementacji pozbawionej błędów formalnych.
System po wykonaniu testów -dodawania tabel, wypełniania ich rekordami działał poprawnie.
Przetestowano moduły zabezpieczeń wspomniane w całym sprawozdaniu projektowym, również
funkcjonowały poprawnie (zabezpieczenia m.in. przed SQLInjection).Projekt całościowo prezentowano
na zajęciach projektowych.\\

W trakcie prac nad realizacją projektu udało się wykorzystać mechanizm replikacji \textit{master-slave} oraz \textit{master-slave} z opóźnieniem. Do ich konfiguracji wykorzystane zostało narzędzie \textit{phpMyAdmin}. \\

Podczas prac nad projektem próbowano wykorzystać jako węzły rozproszonej bazy danych serwery na fizycznych urządzeniach \textit{Raspberry PI}, jednak ilość dostępnych przez nas urządzeń nie pozwalała w pełni pokazać możliwości replikacji w systemie bazodanowym MySQL. Dodatkowo również na system operacyjny \textit{raspbian} nie była dostępna najnowsza wersja MySQL, przez co nie było możliwości m.in. wykonania replikacji z opóźnieniem. Również by replikacja działała wymagane było połączenie internetowe we wszystkich urządzeniach, łącznie z urządzeniem, na którym demonstrowane było działanie. W związku z tym w trakcie prac zrezygnowano z fizycznych urządzeń, a do stworzenia węzłów wykorzystano wirtualne kontenery stworzone w programie \textit{docker}, przez co rozproszona baza danych działa w pełni lokalnie. W każdym kontenerze można było dodać dowolną wersję MySQL oraz \textit{phpMyAdmin}.\\


Interfejs aplikacji bazodanowej jest przejrzysty i intuicyjny, nie jest skomplikowany, dodawanie kolejnych informacji – rekordów nie stanowi żadnych problemów, co jest ważnym atrybutem ze strony spojrzenia konsumenckiego. Implementacja całego projektu pozwoliła na swobodne korzystanie ze strony aktora pracownik jak i aktora klient. Klient posiada inne uprawnienia(na płaszczyźnie szeroko rozumianej rezerwacji seansu w kinie wraz z możliwością opłacenia). Pracownik ma możliwość przeglądania tych rezerwacji, dokonywać ich modyfikacji (np. nagła zmiana repertuaru). Zgodnie z warunkami zadania zaimplementowano możliwość wyświetlania odpowiednich widoków – lista filmów, seansów i biletów.\\

Spełniono wszystkie założenia projektowe postawione przez prowadzącego zajęcia
jak i własne.
	\addcontentsline{toc}{chapter}{Literatura} %utworzenie w spisie treści pozycji Bibliografia
	\pagestyle {empty}

\vspace*{1.3cm}

{\Huge\textbf{Literatura}}

\vspace*{1cm}

\begin{enumerate}[\lbrack 1\rbrack]
	\item Anonim G. i in., \textit{Kronika polska}, T.1: \textit{Kronika i czyny książąt czyli władców polskich}, Gniezno 1115
\end{enumerate}


	
	%\pagestyle {empty}

\vspace*{1.3cm}

{\Huge\textbf{Literatura}}

\vspace*{1cm}

\begin{enumerate}[\lbrack 1\rbrack]
	\item Anonim G. i in., \textit{Kronika polska}, T.1: \textit{Kronika i czyny książąt czyli władców polskich}, Gniezno 1115
\end{enumerate}


	%\bibliography{bibliografia} % wstawia bibliografię korzystając z pliku bibliografia.bib - dotyczy BibTeXa, jeżeli nie korzystamy z BibTeXa należy użyć otoczenia thebibliography
	
\end{document}
