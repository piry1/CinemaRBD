\chapter{Wstęp}

\section{Cele projektu}
Celem projektu jest stworzenie systemu wspomagającego obsługę kina w oparciu o rozproszoną i obiektową bazę danych. System będzie umożliwiać zarządzanie kinem – z wykorzystaniem relacyjnych baz danych replikujących miedzy sobą dane. W pojedynczym węźle bazy danych zawarte będą tabele opisujące miedzy innymi – seanse filmowe, przydział ich do poszczególnych sal kinowych. Aplikacja będzie umożliwiać ponadto tworzenie nowych wpisów w zależności od rodzaju użytkownika obsługującego program. Pracownik kina będzie wprowadzać nowe seanse do bazy; podczas bezpośredniej sprzedażny biletów będzie również wykreślał miejsca
na sali już zajęte – miejsca zawarte na biletach, poszukiwanie rezerwacji wykonanej na konkretna osobę (po imieniu lub nazwisku, czy tez numerze rezerwacji). Użytkownik(Klient) będzie mógł rezerwować konkretne miejsce na określony seans.

\section{Założenia projektowe}
Projekt został wykonany przy użyciu MySQL 5.7. Rozproszoność systemu oparta została o dockery, na których skonfigurowane zostały węzły zarówno slave jak i master. W trakcie realizacji projektu zostały wykorzystane mechanizmy replikacji master-slave oraz master-slave z opóźnieniem. Do wykonania projektu bazy danych wykorzystane zostało narzędzie Microsoft Visio. Zarządzanie bazą danych odbywało się z poziomu narzędzia zwanego phpMyAdmin. Aplikacja kliencka została wykonana w technologii webowej, z wykorzystaniem platformy programistycznej Angular2 oraz języka programowania TypeScript. Komunikacja między bazą danych a aplikacją kliencką zapewnia api restowe napisane w języku PHP. 

\section{Zakres projektu}
Zakres projektu dotyczy zaprojektowania i implementacji rozproszonego systemu bazodanowego dla kina. Projekt składa sę z kilku etapów.

\begin{itemize}
	\item Określenie wymagań funkcjonalnych aplikacji bazodanowej
	\item Testowanie mechanizmów replikacji oraz rozpraszania danych
	\item Opracowanie modelu konceptualnego i fizycznego bazy danych
	\item Implementacja bazy danych, procedur i widoków
	\item Projektowanie i implementacja aplikacji klienckiej
	\item Wdrożenie i testowanie aplikacji klienckiej
\end{itemize}

