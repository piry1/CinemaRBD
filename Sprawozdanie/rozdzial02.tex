\chapter{Replikacja w systemie baz danych MySQL}

\section{Pojęcie replikacji i podstawowe informacje}

Replikacja bazy danych polega na powielaniu bazy danych między różnymi serwerami baz danych, co ma miejsce np. przy pracy w klastrze. Replikacja pozwala na:
\begin{itemize}
	\item \textbf{skalowalność}: dzięki temu możliwe jest rozłożenie obciążenia między wieloma serwerami; operacje zapisu i aktualizacji rekordów mogą odbywać się na jednym serwerze, a pobieranie i przeszukiwanie danych na innych, a znacznie obciążające serwer operacje - na jeszcze innych. Na jednej z kopii mogą pracować analitycy, deweloperzy itp.
	\item \textbf{bezpieczeństwo}: dzięki replikacji tworzymy kopie istniejącej bazy produkcyjnej, które co prawna nie uchronią nas przed operacjami typu \textit{DROP}, ale zapewnią ciągły dostęp do bazy danych w przypadku awarii sprzętu głównego serwera.
	\item \textbf{analizę}: skomplikowane operacje analityczne, różnego rodzaju przeliczenia i analizy statystyczne mogą być wykonywane na osobnym serwerze bez obciążania głównej bazy.
	\item \textbf{separację}: możemy udostępnić kopię bazy produkcyjnej dla deweloperów i testerów, aby swoje prace wykonywali na kopii bazy danych.
	
\end{itemize}

\textbf{Replikację można podzielić na:}
\begin{itemize}
	\item replikację typu \textit{master-slave} - wtedy na bazie produkcyjnej (master) wykonywane są operacje modyfikacji danych, natomiast na pozostałe przenoszona jest kopia bazy danych z serwera głównego
	\item replikację typu \textit{master-master}, inaczej duplikacja, gdzie zmiany czy modyfikacje danych mogą być wykonane na dowolnym komputerze i dochodzi do obustronnej synchronizacji baz danych; dzięki takiemu rozwiązaniu zmiany przeprowadzone na jednej z baz danych zostaną również prowadzone na pozostałych.
\end{itemize}

\section{Replikacja master-slave}

Replikacja danych w \textit{MySQL} opiera się o bardzo prostą zasadę: serwer główny (\textit{master}) prowadzi swego rodzaju dziennik, w którym zapisuje każdą czynność, którą wykonał. Wykorzystuje do tego logi binarne zawierające instrukcje, które wykonał master. Serwer zapasowy (\textit{slave}) odczytuje te dane i kolejno wykonuje zapytania, zapełniając bazę kolejnymi rekordami. Efektem tej pracy są dwie identyczne bazy danych.

Po skonfigurowaniu mechanizmu replikacji na serwerze master pojawia się dodatkowy wątek, który odpowiada za wysyłanie bin-logów do serwerów \textit{slave}. Z kolei serwer zapasowy ma dwa wątki:

\begin{itemize}
	\item \textbf{I/O Thread [wątek wejścia-wyjścia]} - odpowiada za odbieranie dziennika od serwera głównego i zapisuje go w plikach tymczasowych (relay-log),
	\item \textbf{SQL Thread [wątek SQL]} - zajmuje się parsowaniem tych plików i wykonywaniem zapytań do bazy.
\end{itemize}

System bazodanowy MySQL umożliwia trzy różne metody replikacji, co przekłada się na format
danych zapisywanych do bin-logów. Za wybór metody replikacji odpowiada zmienna
binlog\_format, która może przyjąć wartość: ROW, STATEMENT, MIXED. \\

\textbf{Metody replikacji:}
\begin{itemize}
	\item \textbf{SBR (statement-based replication)} - w tym trybie, serwer do pliku zapisuje zapytania jakie
	wykonał.
	\item \textbf{RBR (row-based replication)} - do bin-logów zapisywane są wyniki działań zapytań na
	serwerze master. Zapisywana jest informacja jaki rekord został w jaki sposób zmieniony.
	\item \textbf{MFL (mixed-format logging)} - jest to połączenie dwóch powyższych typów replikacji.
\end{itemize}

Technika replikacji \textbf{SBR} jest bardzo szybka i wydajna, ponieważ w jej przypadku serwer główny
zapisuje do pliku zapytanie jakie wykonał, następnie serwer zapasowy je odczytuje i wykonuje.
Niestety do pliku logów zapisywane są tylko zapytania SQL, co przysporzy nam problemów, gdy
nasze zapytania będą bardziej złożone.\\

Problem ten rozwiązała metoda \textbf{RBR}, która do bin-logów zapisuje wyłącznie zmiany jakie zaszły
po wykonaniu polecenia - logowane są informacje na temat sposobu modyfikacji konkretnych
rekordów. Niestety metoda ta jest znacznie wolniejsza od poprzedniej oraz zwiększa ilość
wysyłanych danych pomiędzy replikującymi się serwerami.\\

Z pomocą przyszła nam metoda \textbf{MFL}, w której w większości przypadków, logowane są zapytania
SQL tak jak w przypadku \textbf{SBR}, natomiast dla zapytań, których wynik nie jest przewidywalny,
włączana jest replikacja \textbf{RBR}.\\


