\chapter{Replikacja w systemie baz danych MySQL}

\section{Pojęcie replikacji i podstawowe informacje}

\section{Replikacja master-slave}

\section{Testowanie mechanizmów replikacji}

W projekcie użyta zostanie konfiguracja Master-Slave. Poniżej wypunktowane zostały wnioski z przeprowadzonych testów replikacji bazodanowej MySql.

\begin{itemize}
	\item Po skonfigurowaniu replikacji wymagane jest utworzenie bazy danych \textit{slave}, która posiada
	tą samą strukturę co \textit{master}
	\item Dane zawarte w bazie \textit{master} nie zostaną automatycznie skopiowane do bazy \textit{slave} po
	skonfigurowaniu replikacji. Należy ręcznie zsynchronizować dane w tabelach.
	\item W przypadku wyłączenia bazy danych \textit{slave} i modyfikacji bazy \textit{master} baza \textit{slave} zostanie
	zsynchronizowana po ponownym podłączeniu do sieci.
	\item \textit{Slave} – odczyt; \textit{master} – zapis, modyfikacja, usuwanie. Gdy \textit{slave} jest wyłączony zapytania
	GET są wysyłane do innego \textit{slave}, a w ostateczności \textit{mastera}. Gdy jest wyłączony \textit{master}
	można jedynie odczytywać dane z serwera. Natomiast na ten czas jakakolwiek modyfikacja
	danych jest niemożliwa.
	\item Od wersji\textit{ MySQL 5.7} możliwa jest replikacja Master-Slave z opóźnieniem. Domyślnie master natychmiastowo wysyła bin-log do węzłów typu \textit{slave}, jednak możliwe jest celowe wprowadzenie opóźnienia, np. w celu ochrony bazy danych przed poleceniem DROP, który wykonany na \textit{masterze}, usunie również bazę/ tabele na standardowych węzłach \textit{slave}. Odpowiednio duże opóźnienie daje możliwość na reakcję ze strony admina, tak aby w razie konieczności ocalić opóźniony węzeł.
\end{itemize}



